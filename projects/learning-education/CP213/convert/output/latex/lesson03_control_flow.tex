\documentclass[11pt]{article}
\usepackage[utf8]{inputenc}
\usepackage{geometry}
\usepackage{amsmath}
\usepackage{amsfonts}
\usepackage{amssymb}
\usepackage{listings}
\usepackage{xcolor}
\usepackage{hyperref}
\usepackage{enumitem}
\usepackage{fancyhdr}
\usepackage{tcolorbox}
\usepackage{graphicx}

% Page setup
\geometry{margin=1in}
\pagestyle{fancy}
\fancyhf{}
\fancyhead[L]{CP213 - Object-Oriented Programming}
\fancyhead[R]{\thepage}
\renewcommand{\headrulewidth}{0.4pt}

% Code listing setup
\lstset{
    language=Java,
    basicstyle=\ttfamily\small,
    keywordstyle=\color{blue},
    commentstyle=\color{green!60!black},
    stringstyle=\color{red},
    numbers=left,
    numberstyle=\tiny,
    stepnumber=1,
    numbersep=5pt,
    backgroundcolor=\color{gray!10},
    showspaces=false,
    showstringspaces=false,
    showtabs=false,
    frame=single,
    tabsize=2,
    captionpos=b,
    breaklines=true,
    breakatwhitespace=false,
    escapeinside={\%*}{*)}
}

% Custom boxes
\newtcolorbox{conceptbox}{
    colback=blue!5,
    colframe=blue!50!black,
    title=Key Concept,
    fonttitle=\bfseries
}

\newtcolorbox{examplebox}{
    colback=green!5,
    colframe=green!50!black,
    title=Example,
    fonttitle=\bfseries
}

\newtcolorbox{warningbox}{
    colback=orange!5,
    colframe=orange!50!black,
    title=Important Note,
    fonttitle=\bfseries
}

\begin{document}

% Title page
\begin{titlepage}
    \centering
    \vspace*{2cm}
    
    {\Huge\bfseries CP213 - Object-Oriented Programming\par}
    \vspace{1cm}
    
    {\Large\bfseries Control Flow in Java\par}
    \vspace{0.5cm}
    
    {\large Lesson 03\par}
    \vspace{2cm}
    
    \begin{tabular}{ll}
        \textbf{Student:} & Solomon Olufelo \\
        \textbf{Student ID:} & 210729170 \\
        \textbf{Email:} & oluf9170@mylaurier.ca \\
        \textbf{Date:} & \today \\
    \end{tabular}
    
    \vfill
    
    {\large Wilfrid Laurier University\par}
    {\large Department of Computer Science\par}
\end{titlepage}

\tableofcontents
\newpage

% Content will be inserted here
\section{Lesson 03: Control Flow in Java}

\subsection{\textbf{1. Introduction}}

\begin{itemize}
\item Code executes \textbf{top to bottom}, unless modified by \textbf{control flow statements}.
\item Control flow allows \textbf{branching} (decisions) and \textbf{looping} (repetition).
\end{itemize}

\subsubsection{\textbf{🔹 Types of Control Flow in Java}}

\begin{itemize}
\item \textbf{Decision-making:} \texttt{if-then}, \texttt{if-then-else}, \texttt{switch}
\item \textbf{Looping:} \texttt{for}, \texttt{while}, \texttt{do-while}
\item \textbf{Branching:} \texttt{break}, \texttt{continue}, \texttt{return}
\end{itemize}

\subsection{\textbf{2. Boolean Expressions & Operators}}

\subsubsection{\textbf{🔹 Comparison Operators}}

\begin{table}[h]
\centering
\begin{tabular}{|c|c|}
\hline
Operator & Meaning \\
\hline
\texttt{==} \\
\hline
\end{tabular}
\caption{Table}
\end{table}
 equal to \begin{table}[h]
\centering
\begin{tabular}{|c|c|c|c|c|c|c|c|c|c|c|c|c|c|c|c|c|c|}
\hline
\texttt{!=} & not equal to & \texttt{>} & greater than & \texttt{>=} & greater than or equal to & \texttt{<} & less than & \texttt{<=} & less than or equal to & ➡️ Remember: \textbf{\texttt{==} for equality}, \textbf{\texttt{=} for assignment}

\textbf{Example:}

\begin{lstlisting}[language=Java, caption=Java Code]
int x = 5, y = 8;
boolean z = x < y;
System.out.println(x < y); // true
\end{lstlisting}

\subsubsection{\textbf{2.2 Comparing Strings}}

\begin{itemize}
\item \texttt{==} → compares memory addresses (not values).
\item Use methods:
\end{itemize}
    - \texttt{str1.equals(str2)}
    - \texttt{str1.equalsIgnoreCase(str2)}

\subsubsection{\textbf{2.3 Lexicographic Ordering}}

\begin{itemize}
\item Based on \textbf{ASCII} values:
\end{itemize}
    - Uppercase letters come before lowercase.
\begin{itemize}
\item Use \texttt{compareTo} methods:
\end{itemize}
    - \texttt{s1.compareTo(s2)} → \texttt{<0}, \texttt{=0}, or \texttt{>0}
    - \texttt{s1.compareToIgnoreCase(s2)} → ignores case differences.

\subsubsection{\textbf{2.4 Building Boolean Expressions}}

\begin{itemize}
\item \textbf{AND (\texttt{&&})} → true only if both are true.
\item \textbf{OR (\texttt{ & })} → true if at least one is true.
\item \textbf{NOT (\texttt{!})} → negates a condition.
\end{itemize}

❌ Invalid:

\begin{lstlisting}[language=Java, caption=Java Code]
min < result < max
\end{lstlisting}

✅ Correct:

\begin{lstlisting}[language=Java, caption=Java Code]
(min < result) && (result < max)
\end{lstlisting}

\subsubsection{\textbf{2.5 Evaluating Boolean Expressions}}

\begin{itemize}
\item Produce \texttt{true} or \texttt{false}.
\item Can assign to variables:
\end{itemize}

\begin{lstlisting}[language=Java, caption=Java Code]
int x = 3, y = 5;
boolean z = x > y; // false
\end{lstlisting}

⚠️ \textbf{Common mistake:}

\begin{lstlisting}[language=Java, caption=Java Code]
boolean b = false;
if (b = true) {   // assignment, not comparison
    System.out.println("Hello"); // executes
}
\end{lstlisting}

\subsubsection{\textbf{2.6 Short-Circuit vs Complete Evaluation}}

\begin{itemize}
\item \textbf{Short-circuit (lazy):}
\end{itemize}
    - \texttt{&&}: stops if first is \texttt{false}.
    - \texttt{ & }: stops if first is \texttt{true}.

✅ Prevents runtime errors:

\begin{lstlisting}[language=Java, caption=Java Code]
int kids = 0, toys = 3;
if ((toys / 3) >= 2 && (toys / kids) >= 2) {
    // safe: second part never runs
}
\end{lstlisting}

\begin{itemize}
\item \textbf{Complete evaluation:}
\end{itemize}
    - Use \texttt{&} and \texttt{ & } instead of \texttt{&&} and \texttt{ & }.
    - Forces evaluation of both sides.

\subsection{\textbf{3. Branching with If-Else Statements}}

\subsubsection{\textbf{3.0 If-Else Basics}}

\begin{itemize}
\item \textbf{Definition}: Lets a program choose between two paths depending on a Boolean condition.
\item \textbf{Syntax}:
\end{itemize}

\begin{lstlisting}[language=Java, caption=Java Code]
if (Boolean_Expression)
    Yes_Statement;
else
    No_Statement;
\end{lstlisting}

\begin{itemize}
\item \textbf{Example}:
\end{itemize}

\begin{lstlisting}[language=Java, caption=Java Code]
int mark = 0;
if (mark > 85)
    System.out.println("Your grade is A");
else
    System.out.println("Your grade is not A");
\end{lstlisting}

👉 \textbf{Java vs Python}:

\begin{itemize}
\item Java requires \textbf{parentheses} around conditions and \textbf{braces \texttt{{ }}} for multiple statements.
\item Python uses \texttt{:} with indentation (\texttt{if ...:}, \texttt{elif ...:}, \texttt{else:}).
\end{itemize}

\subsubsection{\textbf{3.1 Compound Statements}}

\begin{itemize}
\item If a branch has \textbf{multiple statements}, group them in \texttt{{ }}.
\end{itemize}

\begin{lstlisting}[language=Java, caption=Java Code]
if (myScore > yourScore) {
    System.out.println("I win!");
    wager += 100;
} else {
    System.out.println("I wish these were golf scores.");
    wager = 0;
}
\end{lstlisting}

\subsubsection{\textbf{3.2 Omitting Else}}

\begin{itemize}
\item \texttt{else} is optional → creates a simple \texttt{if} statement.
\end{itemize}

\begin{lstlisting}[language=Java, caption=Java Code]
if (weight > ideal)
    calorieIntake -= 500;
\end{lstlisting}

\begin{itemize}
\item With multiple actions, braces are required:
\end{itemize}

\begin{lstlisting}[language=Java, caption=Java Code]
if (weight > ideal) {
    calorieIntake -= 500;
    System.out.println("Increase exercise rate!");
    exerciseRate *= 1.3;
}
\end{lstlisting}

\subsubsection{\textbf{3.3 Nested If-Else}}

\begin{itemize}
\item If statements can be placed \textbf{inside each other}.
\end{itemize}

\begin{lstlisting}[language=Java, caption=Java Code]
if (a > b) {
    if (c > d) {
        if (g < h) {
            // code
        }
    }
} else if (m == n) {
    if (z > y) {
        // code
    }
}
\end{lstlisting}

👉 Use indentation for clarity.

\subsubsection{\textbf{3.4 Multiway If-Else}}

\begin{itemize}
\item Chains multiple conditions.
\end{itemize}

\begin{lstlisting}[language=Java, caption=Java Code]
if (number < 100 && number >= 1) {
    System.out.println("Two-digit number");
} else if (number < 1000 && number >= 100) {
    System.out.println("Three-digit number");
} else if (number < 10000 && number >= 1000) {
    System.out.println("Four-digit number");
} else {
    System.out.println("Number not in range 1–99999");
}
\end{lstlisting}

\subsubsection{\textbf{3.5 Ternary Operator (\texttt{?:})}}

\begin{itemize}
\item Shorthand for simple if-else assignments.
\end{itemize}

\begin{lstlisting}[language=Java, caption=Java Code]
int max = (n1 > n2) ? n1 : n2;
\end{lstlisting}

\begin{itemize}
\item Reads as:
\end{itemize}
    - If condition is \textbf{true}, take first value.
    - If \textbf{false}, take second value.

✅ Summary:

\begin{itemize}
\item \textbf{if-else} controls branching.
\item \textbf{Compound} → multiple statements in \texttt{{ }}.
\item \textbf{Omitting else} → runs only when condition is true.
\item \textbf{Nested if} → allows deeper decisions.
\item \textbf{Multiway if-else} → handles multiple conditions.
\item \textbf{Ternary operator} → compact shorthand.
\end{itemize}

\subsection{\textbf{4. The Switch Statement}}

\subsubsection{\textbf{📌 Purpose}}

\begin{itemize}
\item Implements \textbf{multiway branching} in Java.
\end{itemize}

\subsubsection{\textbf{📝 Syntax}}

\begin{lstlisting}[language=Java, caption=Java Code]
switch (Controlling_Expression) {
    case Case\textit{Label}1:
        // code
        break;
    case Case\textit{Label}2:
        // code
        break;
    default:
        // code if no case matches
        break;
}
\end{lstlisting}

\subsubsection{\textbf{⚖️ Key Rules}}

\begin{itemize}
\item Controlling expression must be: \texttt{char}, \texttt{int}, \texttt{short}, \texttt{byte}, or (Java 7+) \texttt{String}.
\item Each \textbf{case label} must be the same type as the controlling expression.
\item Labels can be in any order but must be \textbf{unique}.
\item \texttt{break;} ends execution of a case.
\item If \texttt{break} is omitted → execution \textbf{falls through} to the next case.
\end{itemize}

\subsubsection{\textbf{4.1 Default Section}}

\begin{itemize}
\item Handles all values \textbf{not explicitly matched} by any case.
\item \textbf{Optional}, but good practice to include.
\item Usually placed \textbf{last}.
\end{itemize}

\subsubsection{\textbf{✅ Example}}

\begin{lstlisting}[language=Java, caption=Java Code]
int numberOfIceCreamFlavors = 15;

switch (numberOfIceCreamFlavors) {
    case 15:
        System.out.println("Nice selection.");
        break;
    case 1:
        System.out.println("I bet it's vanilla.");
        break;
    default:
        System.out.println("Unknown number of flavors.");
        break;
}
\end{lstlisting}

\subsubsection{\textbf{4.1.1 Non-Graded Activity}}

\begin{lstlisting}[language=Java, caption=Java Code]
import java.util.Scanner;

public class SwitchClass {
    public static void main(String[] args) {
        Scanner keyboard = new Scanner(System.in);

        System.out.println("Type a number that represents your lucky day");
        int luckyDay = keyboard.nextInt();

        switch (luckyDay) {
            case 1:
                System.out.println("Your lucky day is Monday");
                break;
            case 2:
                System.out.println("Your lucky day is Tuesday");
                break;
            case 3:
            case 4:
            case 5:
                System.out.println("Your lucky day is Wednesday, Thursday or Friday");
                System.out.println("is acceptable.");
                break;
            default:
                System.out.println("Your lucky day is either Saturday or Sunday");
                break;
        }
    }
}
\end{lstlisting}

\subsection{\textbf{5. Loops in Java}}

Loops allow you to \textbf{repeat a block of code} multiple times. Each repetition is called an \textbf{iteration}, and the code inside the loop is called the \textbf{loop body}.

Java has \textbf{three types of loops}:

1. \texttt{while} loop
2. \texttt{do-while} loop
3. \texttt{for} loop

\subsubsection{\textbf{5.1 while Statement}}

\begin{itemize}
\item Executes the loop \textbf{only if} the Boolean expression is true.
\item \textbf{Condition is checked before} each iteration.
\item Loop body can be \textbf{a single statement} or \textbf{multiple statements enclosed in braces \texttt{{}}}.
\end{itemize}

\textbf{Syntax:}

\begin{lstlisting}[language=Java, caption=Java Code]
while (Boolean_Expression)
    Statement;
\end{lstlisting}

Or with a compound body:

\begin{lstlisting}[language=Java, caption=Java Code]
while (Boolean_Expression) {
    Statement_1;
    Statement_2;
    ...
    Statement_Last;
}
\end{lstlisting}

\subsubsection{\textbf{5.2 do-while Statement}}

\begin{itemize}
\item Executes the loop body \textbf{at least once}.
\item Boolean expression is \textbf{checked after} the loop body executes.
\item Acts like a \texttt{while} loop after the first iteration.
\item Loop body can also be \textbf{single or compound statements}.
\item \textbf{Important:} there is a \textbf{semicolon after the Boolean expression}.
\end{itemize}

\textbf{Syntax:}

\begin{lstlisting}[language=Java, caption=Java Code]
do
    Statement;
while (Boolean_Expression);
\end{lstlisting}

Or with a compound body:

\begin{lstlisting}[language=Java, caption=Java Code]
do {
    Statement_1;
    Statement_2;
    ...
    Statement_Last;
} while (Boolean_Expression);
\end{lstlisting}

\subsubsection{\textbf{5.3 Example: while vs do-while}}

\begin{lstlisting}[language=Java, caption=Java Code]
package lesson03;

public class WhileExample {

    public static void main(String[] args) {
        int countDown;

        System.out.println("First while loop:");
        countDown = 5;
        while (countDown > 0) {
            System.out.println("Hello");
            countDown -= 1;
        }

        System.out.println("Second while loop:");
        countDown = 0;
        while (countDown > 0) { // Won't run
            System.out.println("Hello");
            countDown -= 1;
        }

        System.out.println("First do-while loop:");
        countDown = 5;
        do {
            System.out.println("Hello");
            countDown -= 1;
        } while (countDown > 0);

        System.out.println("Second do-while loop:");
        countDown = 0;
        do { // Executes at least once
            System.out.println("Hello");
            countDown -= 1;
        } while (countDown > 0);
    }
}
\end{lstlisting}

\textbf{Key takeaway:}

\begin{itemize}
\item \texttt{while} may not execute if the condition is false at the start.
\item \texttt{do-while} \textbf{always executes at least once}.
\end{itemize}

\subsection{\textbf{5.4 The \texttt{for} Statement}}

The \texttt{for} loop is commonly used to iterate over a range of values with a control variable.

\textbf{Syntax:}

\begin{lstlisting}[language=Java, caption=Java Code]
for (Initializing; Boolean_Expression; Update) {
    Statement_1;
    Statement_2;
    ...
    Statement_Last;
}
\end{lstlisting}

\begin{itemize}
\item \textbf{Initializing:} Set up control variable(s). Can declare multiple variables separated by commas.
\item \textbf{Boolean_Expression:} Condition checked before each iteration.
\item \textbf{Update:} Changes control variable(s) after each iteration. Can contain multiple updates separated by commas.
\item \textbf{Body:} Single or compound statements.
\end{itemize}

\textbf{Examples:}

\begin{lstlisting}[language=Java, caption=Java Code]
// Single statement
for (int i = 0; i < 3; i++)
    System.out.println("i is " + i);

// Multiple statements
int j = 0;
for (int i = 0; i < 3; i++) {
    j += 1;
    System.out.println("hello " + j);
    System.out.println("j is " + j);
}
\end{lstlisting}

\textbf{Multiple Initialization Example:}

\begin{lstlisting}[language=Java, caption=Java Code]
int j = 0;
for (int i = 0; (i + j) < 6; i++) {
    j++;
    System.out.println("i and j: " + i + " " + j);
}
\end{lstlisting}

\textbf{Multiple Updates Example:}

\begin{lstlisting}[language=Java, caption=Java Code]
for (int i = 0, j = 0; (i + j) < 6; i++, j++, System.out.println("i and j: " + i + " " + j));
\end{lstlisting}

\subsubsection{\textbf{5.4.1 Infinite Loops}}

\begin{itemize}
\item Loops must modify the condition to eventually terminate.
\item Equality (\texttt{==}) and inequality (\texttt{!=}) checks with \texttt{double} or \texttt{float} may cause \textbf{infinite loops} due to precision errors.
\item Example of risky loop:
\end{itemize}

\begin{lstlisting}[language=Java, caption=Java Code]
double x = 1.0 / 3;
for (int i = 0, j = 0; x != 0.33333; i++) {
    j++;
    System.out.println("i and j: " + i + " " + j);
}
\end{lstlisting}

\begin{itemize}
\item A \texttt{for(;;)} statement is a valid infinite loop.
\end{itemize}

\subsubsection{\textbf{5.4.2 Nested Loops}}

\begin{itemize}
\item Loops can be \textbf{nested} like other control structures.
\item Inner loop executes \textbf{completely} for every iteration of the outer loop.
\end{itemize}

\textbf{Example:}

\begin{lstlisting}[language=Java, caption=Java Code]
for (int rowNum = 1; rowNum <= 3; rowNum++) {
    for (int columnNum = 1; columnNum <= 2; columnNum++) {
        System.out.print("row " + rowNum + " column " + columnNum + " ");
    }
    System.out.println();
}
\end{lstlisting}

\subsection{\textbf{5.5 For-each Loop (Enhanced for loop)}}

\begin{itemize}
\item Introduced in \textbf{Java 5} for \textbf{arrays} and \textbf{collections}.
\item Iterates through elements \textbf{without using indices}.
\end{itemize}

\textbf{Syntax:}

\begin{lstlisting}[language=Java, caption=Java Code]
for (type var : arrayOrCollection) {
    body-of-loop;
}
\end{lstlisting}

\textbf{Array Example:}

\begin{lstlisting}[language=Java, caption=Java Code]
double sum = 0;
double[] price = {1.5, 2.5, 3.5, 4, 0.5};

// Basic for loop
for (int i = 0; i < price.length; i++)
    sum += price[i];
System.out.println(sum);

// For-each loop
sum = 0;
for (double p : price)
    sum += p;
System.out.println(sum);
\end{lstlisting}

\textbf{Collection Example:}

\begin{lstlisting}[language=Java, caption=Java Code]
Collection<String> myCollectionItem = new ArrayList<>();
myCollectionItem.add("1 - H - Hydrogen");
myCollectionItem.add("2 - He - Helium");
myCollectionItem.add("3 - Li - Lithium");

System.out.println("Using iterator:");
Iterator<String> iter = myCollectionItem.iterator();
while (iter.hasNext()) {
    System.out.println(iter.next());
}

System.out.println("Using for-each:");
for (String item : myCollectionItem)
    System.out.println(item);
\end{lstlisting}

\textbf{Rules for For-each Loop:}

\begin{itemize}
\item Loop variable \textbf{cannot be assigned} inside the loop.
\item Can \textbf{only iterate one collection} at a time.
\item Iterates \textbf{forward by single steps}.
\end{itemize}

\subsection{\textbf{5.6 The \texttt{break} and \texttt{continue} Statements}}

\subsubsection{\textbf{\texttt{break}}}

\begin{itemize}
\item Syntax: \texttt{break;}
\item Immediately \textbf{terminates the nearest enclosing loop or switch statement}.
\item Control moves to the statement immediately after the loop or switch.
\item Does \textbf{not} terminate the program.
\end{itemize}

\textbf{Example – break in inner loop:}

\begin{lstlisting}[language=Java, caption=Java Code]
for (int i = 0; i < 3; i++) {
    for (int j = 0; j < 3; j++) {
        System.out.println(i + " , " + j);
        break;  // exits inner loop only
    }
}
\end{lstlisting}

\textbf{Output:}

\begin{lstlisting}[language=text, caption=Text Code]
0 , 0
1 , 0
2 , 0
\end{lstlisting}

\subsubsection{\textbf{\texttt{continue}}}

\begin{itemize}
\item Syntax: \texttt{continue;}
\item Ends the \textbf{current iteration} of the nearest enclosing loop.
\item Control moves to the \textbf{next iteration} of the loop.
\item In a \texttt{for} loop, it transfers control to the \textbf{update expression}.
\end{itemize}

\textbf{Example – continue in nested loop:}

\begin{lstlisting}[language=Java, caption=Java Code]
for (int i = 0; i < 3; i++) {
    for (int j = 0; j < 3; j++) {
        if (i >= 1)
            continue;
        System.out.println(i + " , " + j);
    }
}
\end{lstlisting}

\textbf{Output:}

\begin{lstlisting}[language=text, caption=Text Code]
0 , 0
0 , 1
0 , 2
\end{lstlisting}

\subsection{\textbf{5.7 Labeled Break Statement}}

\begin{itemize}
\item Used to \textbf{break out of an outer loop} from an inner loop.
\item Syntax:
\end{itemize}

\begin{lstlisting}[language=Java, caption=Java Code]
labelName: for(...) {
    for(...) {
        break labelName;  // exits the labeled loop
    }
}
\end{lstlisting}

\textbf{Example:}

\begin{lstlisting}[language=Java, caption=Java Code]
endBothLoops:
for (int i = 0; i < 3; i++) {
    for (int j = 0; j < 3; j++) {
        System.out.println(i + " , " + j);
        break endBothLoops;
    }
}
\end{lstlisting}

\textbf{Output:}

\begin{lstlisting}[language=text, caption=Text Code]
0 , 0
\end{lstlisting}

\begin{itemize}
\item Similarly, \texttt{continue} can be used with labels in Java 8+.
\end{itemize}

\subsubsection{\textbf{5.7.1 Program Termination}}

\begin{itemize}
\item \texttt{System.exit(0);} immediately terminates the \textbf{entire program}.
\item The argument \texttt{0} indicates a normal termination.
\end{itemize}

\subsection{\textbf{5.8 Common Loop Errors}}

1. \textbf{Infinite loops:} Condition never becomes false.
2. \textbf{Off-by-one errors:} Loop executes \textbf{one extra or one fewer iteration} than intended.
3. \textbf{Floating-point comparisons:} Avoid \texttt{==} or \texttt{!=} in Boolean expressions for \texttt{float}/\texttt{double}.

\subsection{\textbf{5.9 Java Iterator Interface}}

The \textbf{Iterator} interface in Java provides a standard way to traverse elements in any collection (like \texttt{ArrayList}, \texttt{HashSet}, etc.).

\subsubsection{\textbf{Using a \texttt{for} loop with ArrayList:}}

\begin{lstlisting}[language=Java, caption=Java Code]
ArrayList<String> list = new ArrayList<>();
list.add("Apple");
list.add("Banana");
list.add("Cherry");

for (int i = 0; i < list.size(); i++) {
    String s = list.get(i);
    System.out.println(s);
}
\end{lstlisting}

\subsubsection{\textbf{Using an \texttt{Iterator}:}}

\begin{lstlisting}[language=Java, caption=Java Code]
Iterator<String> itr = list.iterator();
while (itr.hasNext()) {
    String s = itr.next();
    System.out.println(s);
}
\end{lstlisting}

\textbf{Key Points:}

\begin{itemize}
\item \texttt{iterator()} returns an \texttt{Iterator} object for the collection.
\item \texttt{hasNext()} checks if there are more elements to traverse.
\item \texttt{next()} returns the next element in the collection.
\item Using \texttt{Iterator} is preferred when you want to remove elements safely during iteration.
\end{itemize}

\subsection{\textbf{5.10 Assertion Checks}}

An \textbf{assertion} is a statement that tests a condition which should always be true if the program is running correctly.

\subsubsection{\textbf{Syntax:}}

\begin{lstlisting}[language=Java, caption=Java Code]
assert Boolean\textit{Expression;   // Example: assert database}version == 1.2;
\end{lstlisting}

\begin{itemize}
\item If the expression evaluates to \textbf{true}, the program continues normally.
\item If \textbf{false}, the program throws an \texttt{AssertionError} and stops execution.
\item Assertions are mainly used for \textbf{debugging and testing}, not for production logic.
\end{itemize}

\subsubsection{\textbf{Enabling Assertions}}

By default, assertions are \textbf{disabled}. To enable them:

\textbf{Command line:}

\begin{lstlisting}[language=bash, caption=Bash Code]
java -enableassertions ProgramName
\end{lstlisting}

Or using \textbf{Eclipse configuration}, enable assertions in the runtime settings.

\subsubsection{\textbf{Example Use Case:}}

\begin{lstlisting}[language=Java, caption=Java Code]
double database_version = 1.0;
assert database_version == 1.2;  // Throws AssertionError if false
\end{lstlisting}

\begin{itemize}
\item Helps ensure assumptions in your code are valid.
\item Useful in team environments to catch version mismatches or logic errors early.
\end{itemize}

\textbf{Summary Table} & Feature & Description & Example \\
\hline
Iterator \\
\hline
\end{tabular}
\caption{Table}
\end{table}
 Traverses elements in collections \begin{table}[h]
\centering
\begin{tabular}{|c|c|c|c|c|c|c|c|c|c|c|c|c|c|}
\hline
\texttt{itr.next()} & hasNext() & Checks if more elements exist & \texttt{itr.hasNext()} & Assertions & Verifies conditions during runtime & \texttt{assert x > 0;} & Enabling assertions & Must be enabled explicitly & \texttt{java -enableassertions ProgramName} & \subsection{\textbf{6 Generating Random Numbers}}

Java provides the \textbf{\texttt{Random}} class in the \texttt{java.util} package to generate pseudo-random numbers. These are \textbf{not truly random}, but are sufficiently unpredictable for most applications.

\subsubsection{\textbf{Importing the Random Class}}

\begin{lstlisting}[language=Java, caption=Java Code]
import java.util.Random;
\end{lstlisting}

\subsubsection{\textbf{Creating a Random Object}}

\begin{lstlisting}[language=Java, caption=Java Code]
Random rnd = new Random();
\end{lstlisting}

\subsubsection{\textbf{Generating Random Numbers}} & Method & Description & Example \\
\hline
\texttt{nextInt(n)} \\
\hline
\end{tabular}
\caption{Table}
\end{table}
 Random integer from \texttt{0} to \texttt{n-1} \begin{table}[h]
\centering
\begin{tabular}{|c|c|c|c|c|c|c|c|c|c|c|c|c|c|c|c|}
\hline
\texttt{int i = rnd.nextInt(10); // 0-9} & \texttt{nextDouble()} & Random double between \texttt{0.0} (inclusive) and \texttt{1.0} (exclusive) & \texttt{double d = rnd.nextDouble();} & \texttt{nextBoolean()} & Random boolean value (\texttt{true} or \texttt{false}) & \texttt{boolean b = rnd.nextBoolean();} & \texttt{nextLong()} & Random long value & \texttt{long l = rnd.nextLong();} & \texttt{nextFloat()} & Random float between \texttt{0.0} and \texttt{1.0} & \texttt{float f = rnd.nextFloat();} & \subsubsection{\textbf{Example: Simulating a Coin Flip}}

\begin{lstlisting}[language=Java, caption=Java Code]
package lesson03;

import java.util.Random;

public class CoinFlipDemo {

    public static void main(String[] args) {
        Random randomGenerator = new Random();
        int counter = 1;

        while (counter <= 5) {
            System.out.print("Flip number " + counter + ": ");
            int coinFlip = randomGenerator.nextInt(2); // 0 or 1

            if (coinFlip == 1)
                System.out.println("Heads");
            else
                System.out.println("Tails");

            counter++;
        }
    }
}
\end{lstlisting}

\textbf{Explanation:}

\begin{itemize}
\item \texttt{nextInt(2)} generates either \texttt{0} or \texttt{1}.
\item The \texttt{if} statement checks the value to print "Heads" or "Tails".
\item The loop runs 5 times, simulating 5 coin flips.
\end{itemize}

\subsubsection{\textbf{Tips:}}

\begin{itemize}
\item You can experiment with other \texttt{Random} methods like \texttt{nextDouble()}, \texttt{nextLong()}, \texttt{nextBoolean()}, or even generate numbers in a custom range.
\item Random numbers are \textbf{pseudo-random}, meaning if you use the same seed (optional in \texttt{new Random(seed)}), the sequence of numbers is repeatable.
\end{itemize}

\subsection{\textbf{Key Terms Glossary}} & Keyword & Description & Notes \\
\hline
\textbf{assert} \\
\hline
\end{tabular}
\caption{Table}
\end{table}
 Checks a Boolean condition; throws \texttt{AssertionError} if false; mainly used for debugging. |
| \textbf{break} | Exits the nearest enclosing loop or \texttt{switch}; can be labeled to exit outer loops in nested loops. |
| \textbf{continue} | Skips the rest of the current loop iteration; in a \texttt{for} loop, moves to the update expression. |
| \textbf{do-while} | Executes the loop body \textbf{at least once}, then checks the condition; ends if false. |
| \textbf{for-each} | Enhanced loop to iterate arrays or collections; cannot modify loop variable; one element at a time. |
| \textbf{if-else} | Executes one branch if condition is true, another if false; can be nested or multiway (\texttt{else if}). |
| \textbf{iterator} | Interface to traverse collections; \texttt{hasNext()} checks for next element, \texttt{next()} retrieves it. |
| \textbf{java short-circuits} | Logical operators \texttt{&&} and \texttt{\|\|} evaluate left-to-right and stop early if result is determined. |
| \textbf{labeled break} | Break with a label to exit a specific outer loop in nested loops. |
| \textbf{switch} | Multiway branching; executes the case that matches the controlling expression; use \texttt{break} to stop. |
| \textbf{ternary operator} | Compact \texttt{if-else} expression: \texttt{(condition) ? valueIfTrue : valueIfFalse}. |
| \textbf{while} | Executes the loop body \textbf{only if} the condition is true; condition checked before each iteration. |

\subsubsection{\textbf{Notes & Tips:}}

\begin{itemize}
\item \textbf{Break & continue} are best used sparingly; overuse can make loops hard to read.
\item \textbf{For-each} simplifies loops over arrays/collections, but can't modify the elements directly.
\item \textbf{Assertions} are mostly for development/testing, not production.
\item \textbf{Ternary operator} is ideal for short, conditional assignments.
\item \textbf{Infinite loops} often happen with \texttt{while(true)} or incorrect Boolean conditions; watch \texttt{==} with floating-point numbers.
\end{itemize}


\end{document}
